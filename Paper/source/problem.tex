\section{Постановка задачи}

Рассмотрим сферу единичного радиуса с центром в начале координат $\Sigma$. Внутренность этой сферы обозначим $T$.

Пусть внутри сферы $\Sigma$ находятся $N$ точечных электрических источников с зарядами, равными единице. Координаты $i$-го источника обозначим $(x_i, y_i, z_i)$, $i = \overline{1, N}$.

Введём $3N$-вектор $\vect{v}=(x_1,y_1,z_1,x_2,y_2,z_2,...,x_N,y_N,z_N)$, определяющий координаты всех источников, и функцию
\begin{equation}
	f(x,y,z;\vect{v}) = 4\pi \sum_{i = 1}^{N} \delta(x - x_i,y - y_i,z - z_i) \text{, } (x,y,z) \in T \label{f}
\end{equation}

Пусть потенциал электрического поля на поверхности сферы равен нулю. Тогда потенциал электрического поля внутри сферы определяется задачей Дирихле для уравнения Пуассона:
\begin{empheq}[left={\empheqlbrace}]{align}
	&\Delta u(x,y,z) = - f(x,y,z;\vect{v}) \text{, } (x,y,z) \in T\text{, }\label{eqn:direct_problem_1}\\
	&u(x,y,z) = 0 \text{, } (x,y,z) \in \Sigma \text{.}\label{eqn:direct_problem_2}
\end{empheq}

Таким образом, зная положение источников $\vect{v}$ и решив задачу (\ref{eqn:direct_problem_1}), (\ref{eqn:direct_problem_2}), мы можем найти функцию $u(x,y,z)$ и вычислить нормальную производную потенциала
\[\frac{\partial u}{\partial n}(x,y,z)\text{, } (x,y,z) \in \Sigma\]
на поверхности сферы.

Сформулируем \textbf{прямую задачу}.

Задан вектор $\vect{v}$, определяющий положение источников. Требуется найти нормальную производную потенциала $\frac{\partial u}{\partial n}(x,y,z)$ на поверхности сферы.

\newpage
Сформулируем \textbf{обратную задачу 1}.

Пусть положение источников $(x_i,y_i,z_i)\text{, }i=\overline{1,N}$ неизвестно, т.е. неизвестен вектор $\vect{v}$. Требуется определить положения источников, если на поверхности сферы задана нормальная производная решения задачи (\ref{eqn:direct_problem_1}), (\ref{eqn:direct_problem_2}):
\begin{equation}
	\frac{\partial u}{\partial n}(x,y,z)=q(x,y,z) \text{, } (x,y,z) \in \Sigma \text{.} 
\end{equation}

Возможна и другая постановка обратной задачи.

\textbf{Обратная задача 2.}

Пусть положение источников $(x_i,y_i,z_i)\text{, }i=\overline{1,N}$ неизвестно, т.е. неизвестен вектор $\vect{v}$. Требуется определить положения источников, если на части $\Sigma_1$ поверхности сферы задана нормальная производная решения задачи (\ref{eqn:direct_problem_1}), (\ref{eqn:direct_problem_2}):
\begin{equation}
	\frac{\partial u}{\partial n}(x,y,z)=q(x,y,z) \text{, } (x,y,z) \in \Sigma_1 \text{.}
\end{equation}