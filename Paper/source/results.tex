\section{Результаты численного решения обратных задач}

Численный метод и его программная реализация были использованы для численного решения обратных задач.

Вычислительные эксперименты проводились по следующей схеме: выбирались координаты источников, для них решалась прямая задача и находилась нормальная производная на поверхности сферы (или её части $\Sigma_1$). Эта функция принималась за $Q(\phi, \theta)$ и с ней решалась обратная задача. Полученные в результате координаты сравнивались с точными координатами источников.

Для вычислений, в которых нормальную производную требовалось задать с погрешностью, это делалось так (на примере обратной задачи 1): для относительной погрешности
\begin{equation}
    \delta =
    \sqrt{
        \frac
        {\int\limits_{0}^{2\pi}\int\limits_{0}^{\pi}
        \big(
        Q (\phi,\theta) - Q_\delta (\phi, \theta)
        \big)^2
        \sin{\theta} \, d\phi \, d\theta
        }
        {\int\limits_{0}^{2\pi}\int\limits_{0}^{\pi}
        \big(
        Q (\phi,\theta)
        \big)^2
        \sin{\theta} \, d\phi \, d\theta
        }
    }
    \text{,}
\end{equation}
где $Q_\delta(\phi, \theta)$ - производная с погрешностью, выбиралось желаемое значение. Тогда в ячейке сетки $\Delta \Sigma_{ij}$ к значению производной в центральной точке прибавлялось возмущение:
\begin{equation}
    Q_{ij}(\delta) = \frac{\lambda_{ij}}{\sqrt{\sum_{i,j} \lambda_{ij}^2}} \cdot \sqrt{\frac{\delta}{S_{ij}}} \text{,}
\end{equation}
где $\lambda_{ij}$ выбиралось один раз для каждой ячейки сетки как значение случайной величины с распределением $U[-1;1]$, $S_{ij}$ - площадь данной ячейки сетки.

Перейдём к описанию проведённых вычислительных экспериментов.

% Эксперимент 1
\emph{Эксперимент 1.} Два источника, полная сфера.

Критерий остановки: $\varepsilon = 1e-7$.

Точные (декартовы) координаты источников:
\begin{align*}
    \begin{matrix*}[r]
    (0.7, & 0.7, & 0.0) \text{,}\\
    (0.8, & -0.59, & 0.0) \text{.}
    \end{matrix*}
\end{align*}

Начальные приближения (здесь и далее найдены автоматически):
\begin{align*}
    \begin{matrix*}[r]
    (0.0, & 0.5, & 0.0) \text{,}\\
    (0.5, & 0.0, & 0.0) \text{.}
    \end{matrix*}
\end{align*}

Координаты после 5 итерации:
\begin{align*}
    \begin{matrix*}[r]
    (0.6738, & 0.6642, & -0.0000) \text{,}\\
    (0.7236, & -0.3514, & -0.0001) \text{.}
    \end{matrix*}
\end{align*}

Вычисленные координаты (за 28 итераций):
\begin{align*}
    \begin{matrix*}[r]
    (0.6999, & 0.6999, & 0.0000) \text{,}\\
    (0.7959, & -0.5880, & -0.0005) \text{.}
    \end{matrix*}
\end{align*}

% Эксперимент 2
\emph{Эксперимент 2.} Три источника, полная сфера.

Критерий остановки: $\varepsilon = 1e-7$.

Точные (декартовы) координаты источников:
\begin{align*}
    \begin{matrix*}[r]
    (\phantom{-}0.4, & 0.8, & 0.1) \text{,}\\
    (-0.2, & -0.5, & -0.5) \text{,}\\
    (-0.7, & -0.3, & -0.4) \text{.}
    \end{matrix*}
\end{align*}

Начальные приближения:
\begin{align*}\begin{matrix*}[r]
    (-0.5, & 0.0, & 0.0) \text{,}\\
    (0.0, & 0.5, & 0.0) \text{,}\\
    (0.0, & 0.0, & -0.5) \text{.}
\end{matrix*}\end{align*}

Координаты после 10 итерации:
\begin{align*}
    \begin{matrix*}[r]
    (0.2087, & 0.5378, & 0.0489) \text{,}\\
    (-0.5687, & -0.3294, & -0.3800) \text{,}\\
    (-0.5357, & -0.2798, & -0.3555) \text{.}
    \end{matrix*}
\end{align*}

Вычисленные координаты (за 131 итерацию):
\begin{align*}
    \begin{matrix*}[r]
    (0.3905, & 0.8032, & 0.0970) \text{,}\\
    (-0.1999, & -0.5000, & -0.4999) \text{,}\\
    (-0.7000, & -0.3000, & -0.4000) \text{.}
    \end{matrix*}
\end{align*}

% Эксперимент 3
\emph{Эксперимент 3.} Два источника, сравнение полной сферы с полусферой $z > 0$.

Критерий остановки: $\varepsilon = 1e-4$.

Точные (декартовы) координаты источников:
\begin{align*}
    \begin{matrix*}[r]
    (0.3, & 0.4, & 0.6) \text{,}\\
    (-0.7, & -0.2, & 0.1) \text{.}
    \end{matrix*}
\end{align*}

Начальные приближения (совпали для сферы и полусферы):
\begin{align*}
    \begin{matrix*}[r]
    (0.0, & 0.0, & 0.5) \text{,}\\
    (-0.5, & 0.0, & 0.0) \text{.}
    \end{matrix*}
\end{align*}

Координаты после 5 итерации:

Полная сфера:
\begin{align*}
    \begin{matrix*}[r]
    (0.2270, & 0.0352, & 0.5257) \text{,}\\
    (-0.7122, & -0.1073, & 0.0421) \text{.}
    \end{matrix*}
\end{align*}

Полусфера $z > 0$:
\begin{align*}
    \begin{matrix*}[r]
    (0.2005, & 0.0539, & 0.4026) \text{,}\\
    (-0.7140, & -0.1780, & 0.0653) \text{.}
    \end{matrix*}
\end{align*}

Вычисленные координаты для всей сферы (за 31 итерацию):
\begin{align*}
    \begin{matrix*}[r]
    (0.3069, & 0.3981, & 0.6020) \text{,}\\
    (-0.6999, & -0.1999, & 0.1000) \text{.}
    \end{matrix*}
\end{align*}

Вычисленные координаты для полусферы $z > 0$ (за 34 итерации):
\begin{align*}
    \begin{matrix*}[r]
    (0.3099, & 0.3648, & 0.5777) \text{,}\\
    (-0.6999, & -0.1998, & 0.0996) \text{.}
    \end{matrix*}
\end{align*}

Можно заметить, что в случае неполной сферы полученное приближение оказалось менее точным.

% Эксперимент 4
\emph{Эксперимент 4.} Три источника, сравнение полной сферы с полусферой $x > 0$.

Критерий остановки: $\varepsilon = 1e-5$.

Точные (декартовы) координаты источников:
\begin{align*}
    \begin{matrix*}[r]
    (0.7, & 0.4, & 0.55) \text{,}\\
    (0.1, & -0.8, & 0.3) \text{,}\\
    (0.6, & -0.1, & -0.6) \text{.}
    \end{matrix*}
\end{align*}

Начальные приближения (совпали для сферы и полусферы):
\begin{align*}
    \begin{matrix*}[r]
    (0.0, & 0.0, & 0.5) \text{,}\\
    (0.0, & -0.5, & 0.0) \text{,}\\
    (0.5, & 0.0, & 0.0) \text{.}
    \end{matrix*}
\end{align*}

Координаты после 10 итерации:

Полная сфера:
\begin{align*}
    \begin{matrix*}[r]
    (0.4600, & 0.0892, & 0.4064) \text{,}\\
    (0.0921, & -0.7376, & 0.1786) \text{,}\\
    (0.4044, & 0.0929, & -0.1032) \text{.}
    \end{matrix*}
\end{align*}

Полусфера $x > 0$:
\begin{align*}
    \begin{matrix*}[r]
    (0.7157, & 0.3589, & 0.4639) \text{,}\\
    (0.0325, & -0.7112, & 0.1355) \text{,}\\
    (0.2486, & 0.0131, & -0.0361) \text{.}
    \end{matrix*}
\end{align*}

Вычисленные координаты для всей сферы (за 55 итераций):
\begin{align*}
    \begin{matrix*}[r]
    (0.7044, & 0.3974, & 0.5519) \text{,}\\
    (0.1000, & -0.8000, & 0.3000) \text{,}\\
    (0.6000, & -0.0999, & -0.5999) \text{.}
    \end{matrix*}
\end{align*}

Вычисленные координаты для полусферы $x > 0$ (за 38 итераций):
\begin{align*}
    \begin{matrix*}[r]
    (0.7157, & 0.3589, & 0.4639) \text{,}\\
    (0.0999, & -0.8000, & 0.2999) \text{,}\\
    (0.5999, & -0.1001, & -0.6001) \text{.}
    \end{matrix*}
\end{align*}

Аналогично случаю двух источников, для неполной сферы полученное приближение оказалось менее точным.

% Эксперимент 5
\emph{Эксперимент 5.} Два источника, полная сфера, погрешность в нормальной производной.

Критерий остановки: $\varepsilon = 1e-6$.

Точные (декартовы) координаты источников:
\begin{align*}
    \begin{matrix*}[r]
(-0.4, & -0.65, & 0.1) \text{,}\\
(0.59, & 0.2, & -0.3) \text{.}
    \end{matrix*}
\end{align*}

Начальные приближения:
\begin{align*}
    \begin{matrix*}[r]
(0.0, & -0.5, & 0.0) \text{,}\\
(0.5, & 0.0, & 0.0) \text{.}
    \end{matrix*}
\end{align*}

Вычисленные координаты для данных без погрешности (за 62 итерации):
\begin{align*}
    \begin{matrix*}[r]
(-0.4000, & -0.6501, & 0.1000) \text{,}\\
(0.5900, & 0.1999, & -0.3000) \text{.}
    \end{matrix*}
\end{align*}

Вычисленные координаты для данных с погрешностью $\delta=0.01$ (за 52 итерации):
\begin{align*}
    \begin{matrix*}[r]
(-0.3998, & -0.6497, & 0.0999) \text{,}\\
(0.5900, & 0.1999, & -0.2999) \text{.}
    \end{matrix*}
\end{align*}

Вычисленные координаты для данных с погрешностью $\delta=0.5$ (за 63 итерации):
\begin{align*}
    \begin{matrix*}[r]
(-0.3995, & -0.6500, & 0.1001) \text{,}\\
(0.5901, & 0.1994, & -0.3001) \text{.}
    \end{matrix*}
\end{align*}

Можно видеть, что при достаточно малых $\delta$ метод даёт решения, близкие к решению для точной нормальной производной.

% Эксперимент 6
\emph{Эксперимент 6.} Три источника, полная сфера, погрешность в нормальной производной.

Критерий остановки: $\varepsilon = 1e-5$.

Точные (декартовы) координаты источников:
\begin{align*}
    \begin{matrix*}[r]
    (0.57, & 0.1, & -0.7) \text{,}\\
    (-0.4, & 0.6, & 0.12) \text{,}\\
    (-0.25, & -0.69, & 0.15) \text{.}
    \end{matrix*}
\end{align*}

Начальные приближения:
\begin{align*}
    \begin{matrix*}[r]
    (0.0, & 0.0, & -0.5) \text{,}\\
    (0.0, & 0.5, & 0.0) \text{,}\\
    (0.0, & -0.5, & 0.0) \text{.}
    \end{matrix*}
\end{align*}

Вычисленные координаты для данных без погрешности (за 52 итераций):
\begin{align*}
    \begin{matrix*}[r]
    (0.5830, & 0.0973, & -0.7174) \text{,}\\
    (-0.3998, & 0.6000, & 0.1198) \text{,}\\
    (-0.2499, & -0.6900, & 0.1499) \text{.}
    \end{matrix*}
\end{align*}

Вычисленные координаты для данных с погрешностью $\delta=0.01$ (за 52 итерации):
\begin{align*}
    \begin{matrix*}[r]
    (0.5846, & 0.0976, & -0.7192) \text{,}\\
    (-0.3998, & 0.6000, & 0.1198) \text{,}\\
    (-0.2499, & -0.6900, & 0.1498) \text{.}
    \end{matrix*}
\end{align*}

Вычисленные координаты для данных с погрешностью $\delta=0.5$ (за 61 итерациию):
\begin{align*}
    \begin{matrix*}[r]
    (0.5724, & 0.0817, & -0.7003) \text{,}\\
    (-0.4005, & 0.5996, & 0.1215) \text{,}\\
    (-0.2501, & -0.6889, & 0.1502) \text{.}
    \end{matrix*}
\end{align*}

Аналогично случаю двух источников, при достаточно малых $\delta$ метод даёт решения, близкие к решению для точной нормальной производной.