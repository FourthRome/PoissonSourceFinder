\section{Результаты численного решения обратных задач}

Численный метод и его программная реализация были использованы для численного решения обратных задач.

Вычислительные эксперименты проводились по следующей схеме: выбирались координаты источников, для них решалась прямая задача и находилась нормальная производная на поверхности сферы (или её части $\Sigma_1$). Эта функция принималась за $Q(\phi, \theta)$ и с ней решалась обратная задача. Полученные в результате координаты сравнивались с точными координатами источников.

Перейдём к описанию проведённых вычислительных экспериментов.

% Эксперимент 1
\emph{Эксперимент 1.} Два источника, полная сфера.

Критерий остановки: $\varepsilon = 1e-7$.

Точные (декартовы) координаты источников:
\begin{align}
    (0.7,\ 0.7,\ 0) \text{,}\nonumber\\
    (0.8,\ -0.59,\ 0) \text{.}\nonumber
\end{align}

Начальные приближения (здесь и далее найдены автоматически):
\begin{align}
    (0,\ 0.5,\ 0) \text{,}\nonumber\\
    (0.5,\ 0,\ 0) \text{.}\nonumber
\end{align}

Координаты после 5 итерации:
\begin{align}
    (0.6738,\ 0.6642,\ -0.0000) \text{,}\nonumber\\
    (0.7236,\ -0.3514,\ -0.0001) \text{.}\nonumber
\end{align}

Вычисленные координаты (за 28 итераций):
\begin{align}
    (0.6999,\ 0.6999,\ 0.0000) \text{,}\nonumber\\
    (0.7959,\ -0.5880,\ -0.0005) \text{.}\nonumber
\end{align}

\newpage
% Эксперимент 2
\emph{Эксперимент 2.} Три источника, полная сфера.

Критерий остановки: $\varepsilon = 1e-7$.

Точные (декартовы) координаты источников:
\begin{align}
    (0.4,\ 0.8,\ 0.1) \text{,}\nonumber\\
    (-0.2,\ -0.5,\ -0.5) \text{,}\nonumber\\
    (-0.7,\ -0.3,\ -0.4) \text{.}\nonumber
\end{align}

Начальные приближения:
\begin{align}
    (-0.5,\ 0,\ 0) \text{,}\nonumber\\
    (0,\ 0.5,\ 0) \text{,}\nonumber\\
    (0,\ 0,\ -0.5) \text{.}\nonumber
\end{align}

Координаты после 10 итерации:
\begin{align}
    (0.2087,\ 0.5378,\ 0.0489) \text{,}\nonumber\\
    (-0.5687,\ -0.3294,\ -0.3800) \text{,}\nonumber\\
    (-0.5357,\ -0.2798,\ -0.3555) \text{.}\nonumber
\end{align}

Вычисленные координаты (за 131 итерацию):
\begin{align}
    (0.3905,\ 0.8032,\ 0.0970) \text{,}\nonumber\\
    (-0.1999,\ -0.5000,\ -0.4999) \text{,}\nonumber\\
    (-0.7000,\ -0.3000,\ -0.4000) \text{.}\nonumber
\end{align}

% Эксперимент 3
\emph{Эксперимент 3.} Два источника, сравнение полной сферы с полусферой $z > 0$.

Критерий остановки: $\varepsilon = 1e-4$.

Точные (декартовы) координаты источников:
\begin{align}
    (0.3,\ 0.4,\ 0.6) \text{,}\nonumber\\
    (-0.7,\ -0.2,\ 0.1) \text{.}\nonumber
\end{align}

Начальные приближения (совпали для сферы и полусферы):
\begin{align}
    (0,\ 0,\ 0.5) \text{,}\nonumber\\
    (-0.5,\ 0,\ 0) \text{.}\nonumber
\end{align}

Координаты после 5 итерации:

Полная сфера:
\begin{align}
    (0.2270,\ 0.0352,\ 0.5257) \text{,}\nonumber\\
    (-0.7122,\ -0.1073,\ 0.0421) \text{.}\nonumber
\end{align}

Полусфера $z > 0$:
\begin{align}
    (0.2005,\ 0.0539,\ 0.4026) \text{,}\nonumber\\
    (-0.7140,\ -0.1780,\ 0.0653) \text{.}\nonumber
\end{align}

\newpage
Вычисленные координаты для всей сферы (за 31 итерацию):
\begin{align}
    (0.3069,\ 0.3981,\ 0.6020) \text{,}\nonumber\\
    (-0.6999,\ -0.1999,\ 0.1000) \text{.}\nonumber
\end{align}

Вычисленные координаты для полусферы $z > 0$ (за 34 итерации):
\begin{align}
    (0.3099,\ 0.3648,\ 0.5777) \text{,}\nonumber\\
    (-0.6999,\ -0.1998,\ 0.0996) \text{.}\nonumber
\end{align}

Можно заметить, что в случае неполной сферы полученное приближение оказалось менее точным.

% Эксперимент 4
\emph{Эксперимент 4.} Три источника, сравнение полной сферы с полусферой $x > 0$.

Критерий остановки: $\varepsilon = 1e-5$.

Точные (декартовы) координаты источников:
\begin{align}
    (0.7,\ 0.4,\ 0.55) \text{,}\nonumber\\
    (0.1,\ -0.8,\ 0.3) \text{,}\nonumber\\
    (0.6,\ -0.1,\ -0.6) \text{.}\nonumber
\end{align}

Начальные приближения (совпали для сферы и полусферы):
\begin{align}
    (0,\ 0,\ 0.5) \text{,}\nonumber\\
    (0,\ -0.5,\ 0) \text{,}\nonumber\\
    (0.5,\ 0,\ 0) \text{.}\nonumber
\end{align}

Координаты после 10 итерации:

Полная сфера:
\begin{align}
    (0.4600,\ 0.0892,\ 0.4064) \text{,}\nonumber\\
    (0.0921,\ -0.7376,\ 0.1786) \text{,}\nonumber\\
    (0.4044,\ 0.0929,\ -0.1032) \text{.}\nonumber
\end{align}

Полусфера $x > 0$:
\begin{align}
    (0.7157,\ 0.3589,\ 0.4639) \text{,}\nonumber\\
    (0.0325,\ -0.7112,\ 0.1355) \text{,}\nonumber\\
    (0.2486,\ 0.0131,\ -0.0361) \text{.}\nonumber
\end{align}

\newpage
Вычисленные координаты для всей сферы (за 55 итераций):
\begin{align}
    (0.7044,\ 0.3974,\ 0.5519) \text{,}\nonumber\\
    (0.1000,\ -0.8000,\ 0.3000) \text{,}\nonumber\\
    (0.6000,\ -0.0999,\ -0.5999) \text{.}\nonumber
\end{align}

Вычисленные координаты для полусферы $x > 0$ (за 38 итераций):
\begin{align}
    (0.7157,\ 0.3589,\ 0.4639) \text{,}\nonumber\\
    (0.0999,\ -0.8000,\ 0.2999) \text{,}\nonumber\\
    (0.5999,\ -0.1001,\ -0.6001) \text{.}\nonumber
\end{align}

Аналогично случаю двух источников, для неполной сферы полученное приближение оказалось менее точным.

% Эксперимент 5
\emph{Эксперимент 5.} Два источника, полная сфера, погрешность в нормальной производной.

Критерий остановки: $\varepsilon = 1e-6$.

Точные (декартовы) координаты источников:
\begin{align}
(-0.4,\ -0.65,\ 0.1) \text{,}\nonumber\\
(0.59,\ 0.2,\ -0.3) \text{.}\nonumber
\end{align}

Начальные приближения:
\begin{align}
(0,\ -0.5,\ 0) \text{,}\nonumber\\
(0.5,\ 0,\ 0) \text{.}\nonumber
\end{align}

Вычисленные координаты для данных без погрешности (за 62 итерации):
\begin{align}
(-0.4000,\ -0.6501,\ 0.1000) \text{,}\nonumber\\
(0.5900,\ 0.1999,\ -0.3000) \text{.}\nonumber
\end{align}

Вычисленные координаты для данных с погрешностью $\delta=2$ (за 14 итераций):
\begin{align}
(-0.3995,\ -0.6501,\ 0.1000) \text{,}\nonumber\\
(0.5904,\ 0.2004,\ -0.2995) \text{.}\nonumber
\end{align}

Начальные приближения для данных с погрешностью $\delta=20$ оказались другими:
\begin{align}
(0,\ -0.5,\ 0) \text{,}\nonumber\\
(0,\ 0,\ -0.5) \text{.}\nonumber
\end{align}

Вычисленные координаты для данных с погрешностью $\delta=20$ (за 9 итераций):
\begin{align}
(-0.4013,\ -0.6505,\ 0.1021) \text{,}\nonumber\\
(0.0050,\ 0.0000,\ -0.4334) \text{.}\nonumber
\end{align}

Можно видеть, что при достаточно малых $\delta$ метод даёт решения, близкие к решению для точной нормальной производной.

% Эксперимент 6
\emph{Эксперимент 6.} Три источника, полная сфера, погрешность в нормальной производной.

Критерий остановки: $\varepsilon = 1e-6$.

Точные (декартовы) координаты источников:
\begin{align}
    () \text{,}\nonumber\\
    () \text{,}\nonumber\\
    () \text{.}\nonumber
\end{align}

Начальные приближения:
\begin{align}
    () \text{,}\nonumber\\
    () \text{,}\nonumber\\
    () \text{.}\nonumber
\end{align}

Вычисленные координаты для данных без погрешности (за  итераций):
\begin{align}
    () \text{,}\nonumber\\
    () \text{,}\nonumber\\
    () \text{.}\nonumber
\end{align}

Вычисленные координаты для данных с погрешностью $\delta=1$ (за  итераций):
\begin{align}
    () \text{,}\nonumber\\
    () \text{,}\nonumber\\
    () \text{.}\nonumber
\end{align}

Вычисленные координаты для данных с погрешностью $\delta=20$ (за  итерации):
\begin{align}
    () \text{,}\nonumber\\
    () \text{,}\nonumber\\
    () \text{.}\nonumber
\end{align}

Аналогично случаю двух источников, при достаточно малых $\delta$ метод даёт решения, близкие к решению для точной нормальной производной.