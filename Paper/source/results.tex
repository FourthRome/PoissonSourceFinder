\section{Результаты численного решения обратных задач}

Численный метод и его программная реализация были использованы для численного решения обратных задач.

Вычислительные эксперименты проводились по следующей схеме: выбирались координаты источников, для них решалась прямая задача и находилась нормальная производная на поверхности сферы (или её части $\Sigma_1$). Эта функция принималась за $Q(\phi, \theta)$ и с ней решалась обратная задача. Полученные в результате координаты сравнивались с исходными координатами источников.

Перейдём к описанию проведённых вычислительных экспериментов. Начнём с обратной задачи 1.

\begin{enumerate}
	\item Число источников равно двум.
	
	\begin{enumerate}
		\item 
	\end{enumerate}
\end{enumerate}

Интегралы  методом средних прямоугольников по фиксированной прямоугольной сетке с шагами $0.01$ по переменным $\phi$ и $\theta$.

Для проведения вычислений было написано консольное приложение на языке C\# для платформы .NET Framework.
Результаты вычислительных экспериментов представлены в таблицах ниже.
\begin{center}
	\begin{tabular}{|| >{\centering\arraybackslash}m{3cm} | >{\centering\arraybackslash}m{2cm} | >{\centering\arraybackslash}m{2cm} | >{\centering\arraybackslash}m{2cm} | >{\centering\arraybackslash}m{2cm} | >{\centering\arraybackslash}m{2cm} | >{\centering\arraybackslash}m{2cm} ||}
		\multicolumn{7}{c}{Табл. 2. Результаты для $N = 2$}\\
		\hline
		№ & 7 & 8 & 9 & 10 & 11 & 12\\
		\hline
		Область задания $\phi$ & $[0;2\pi]$ & $[0;2\pi]$ & $[0;2\pi]$ & $[0;\pi]$ & $[0;\pi]$ & $[\frac{3\pi}{8}; \frac{3\pi}{8} + 0.02]$  \\
		\hline
		Область задания $\theta$ & $[0;\pi]$ & $[0;\pi]$ & $[0;\pi]$ & $[\frac{\pi}{2};\pi]$ & $[\frac{\pi}{2};\pi]$ & $[\frac{\pi}{2}; \frac{\pi}{2} + 0.02]$ \\
		\hline
		\multicolumn{7}{||c||}{Реальные коорд. источников}\\
		\hline
		$\rho_1$ & 0.5 & 0.5 & 0.5 & 0.5 & 0.5 & 0.5 \\
		\hline
		$\phi_1$ & 0 & 0 & 0 & 0 & 0 & 0 \\
		\hline
		$\theta_1$ & 0.78539 & 0.78539 & 0.78539 & 0.78539 & 0.78539 & 0.78539 \\
		\hline
		$\rho_2$ & 0.3 & 0.3 & 0.3 & 0.3 & 0.3 & 0.3 \\
		\hline
		$\phi_2$ & 3.14159 & 3.14159 & 3.14159 & 3.14159 & 3.14159 & 3.14159\\
		\hline
		$\theta_2$ & 1.17809 & 1.17809 & 1.17809 & 1.17809 & 1.17809 & 1.17809\\
		\hline
		\multicolumn{7}{||c||}{Вычисленные коорд. источников}\\
		\hline
		$\rho_1$ &  0.46611 & 0.49896 & 0.50147 & 0.37901 & 0.34974 &  0.49100 \\
		\hline
		$\phi_1$ & 6.28318 & 6.28318 & 6.28318 & 5.93112 & 5.90464 & 0.00825\\
		\hline
		$\theta_1$ &  0.99658 & 0.83015 & 0.77998 & 0.90908 & 0.89813 & 1.57088\\
		\hline
		$\rho_2$  & 0.29900 & 0.29999 & 0.29733 & 0.33356 & 0.33408 & 0.49393\\
		\hline
		$\phi_2$ & 3.14159 & 3.14159 & 3.14159 & 3.27688 & 3.28226 & 3.13982\\
		\hline
		$\theta_2$ & 1.37226 & 1.32634 & 1.24389 & 1.14345 & 1.11704 & 1.57081\\
		\hline
		Кол-во итераций & 9 & 13 & 30 & 391 & 389 & 14287\\
		\hline
	\end{tabular}
\end{center}

\begin{center}
	\begin{tabular}{|| >{\centering\arraybackslash}m{3cm} | >{\centering\arraybackslash}m{2cm} | >{\centering\arraybackslash}m{2cm} ||}
		\multicolumn{3}{c}{Табл. 3. Результаты для $N = 3$}\\
		\hline
		№ & 13 & 14\\
		\hline
		Область задания $\phi$ & $[0;2\pi]$ & $[0;\pi]$  \\
		\hline
		Область задания $\theta$ & $[0;\pi]$ & $[\frac{\pi}{2};\pi]$ \\
		\hline
		\multicolumn{3}{||c||}{Реальные коорд. источников}\\
		\hline
		$\rho_1$ & 0.7 & 0.7  \\
		\hline
		$\phi_1$ & 2 & 2 \\
		\hline
		$\theta_1$ & $\frac{\pi}{4}$ & $\frac{\pi}{4}$  \\
		\hline
		$\rho_2$ & 0.3 & 0.3 \\
		\hline
		$\phi_2$ & $\pi$ & $\pi$ \\
		\hline
		$\theta_2$ & $\frac{3\pi}{8}$ & $\frac{3\pi}{8}$  \\
		\hline
		$\rho_3$ & 0.5 & 0.5  \\
		\hline
		$\phi_3$ & 4 & 4   \\
		\hline
		$\theta_3$ & 0 & 0   \\
		
		\hline
		\multicolumn{3}{||c||}{Вычисленные коорд. источников}\\
		\hline
		$\rho_1$  & 0.69312 & 0.33746  \\
		\hline
		$\phi_1$ & 1.97459 & 6.04894  \\
		\hline
		$\theta_1$ & 0.77399 & 1.33579  \\
		\hline
		$\rho_2$ & 0.15753 & 0.34589  \\
		\hline
		$\phi_2$ & 4.04814 & 2.21901  \\
		\hline
		$\theta_2$ & 0.97211 & 1.32105  \\
		\hline
		$\rho_3$ & 0.45629 & 0.70875  \\
		\hline
		$\phi_3$ & 2.92537 & 4.19473  \\
		\hline
		$\theta_3$ & 0.41817 & 1.53936  \\
		\hline
		Погрешность функц. $\varepsilon$ & 0.01 & 0.01\\
		\hline
		Кол-во итераций & 105 & 7 \\
		\hline
		$h_\phi$ & 0.01 & 0.01 \\
		\hline
		$h_\theta$ & 0.01 & 0.01\\
		\hline
	\end{tabular}
\end{center}