\section{Результаты численного решения обратных задач}

Численный метод и его программная реализация были использованы для численного решения обратных задач.

Вычислительные эксперименты проводились по следующей схеме: выбирались координаты источников, для них решалась прямая задача и находилась нормальная производная на поверхности сферы (или её части $\Sigma_1$). Эта функция принималась за $Q(\phi, \theta)$ и с ней решалась обратная задача. Полученные в результате координаты сравнивались с исходными координатами источников.

Перейдём к описанию проведённых вычислительных экспериментов.

\emph{Эксперимент 1.} Два источника, полная сфера.

Точные (декартовы) координаты источников:
\begin{align}
    A_1 =& (0.7, 0.7, 0) \text{,}\nonumber\\
    A_2 =& (0.8, -0.59, 0) \text{.}\nonumber
\end{align}

Начальные приближения (здесь и далее найдены автоматически):
\begin{align}
    B_1 =& (0, 0.5, 0) \text{.}\nonumber\\
    B_2 =& (0.5, 0, 0) \text{,}\nonumber
\end{align}

Вычисленные координаты (за 50 итераций):
\begin{align}
    C_1 =& (0.7000000001093324, 0.7000000007897016, 2.368219107952005E-10) \text{,}\nonumber\\
    C_2 =& (0.796267579377084, -0.5882665798262678, -0.0005869050160635728) \text{,}\nonumber
\end{align}

\newpage
\emph{Эксперимент 2.} Три источника, полная сфера.

Точные (декартовы) координаты источников:
\begin{align}
    A_1 =& (0.4, 0.8, 0.1) \text{,}\nonumber\\
    A_2 =& (-0.2, -0.5, -0.5) \text{,}\nonumber\\
    A_3 =& (-0.7, -0.3, -0.4) \text{.}\nonumber
\end{align}

Начальные приближения:
\begin{align}
    B_1 =& (0, 0.5, 0) \text{,}\nonumber\\
    B_2 =& (-0.5, 0, 0) \text{,}\nonumber\\
    B_3 =& (0, 0, -0.5) \text{.}\nonumber
\end{align}

Вычисленные координаты (за 131 итерацию):
\begin{align}
    C_1 =& (0.3905281573816906, 0.803273929639111, 0.0970822309925765) \text{,}\nonumber\\
    C_2 =& (-0.19998679695050278, -0.5000152733078824, -0.49999710779046874) \text{,}\nonumber\\
    C_3 =& (-0.7000005563132058, -0.30000231485465834, -0.4000008392847207) \text{.}\nonumber
\end{align}

\emph{Эксперимент 3.} Два источника, сравнение полной сферы с полусферой $z > 0$.

Точные (декартовы) координаты источников:
\begin{align}
    A_1 =& (0.3, 0.4, 0.6) \text{,}\nonumber\\
    A_2 =& (-0.7, -0.2, 0.1) \text{.}\nonumber
\end{align}

Начальные приближения (совпали для сферы и полусферы):
\begin{align}
    B_1 =& (0, 0, 0.5) \text{,}\nonumber\\
    B_2 =& (-0.5, 0, 0) \text{.}\nonumber
\end{align}

\newpage
Вычисленные координаты для всей сферы (за 31 итерацию):
\begin{align}
    C_1 =& (0.3069848674845314, 0.3981937729093566, 0.6020560955303462) \text{,}\nonumber\\
    C_2 =& (-0.69999415021141, -0.1999516129307181, 0.10004786228560586) \text{.}\nonumber
\end{align}

Вычисленные координаты для полусферы $z > 0$ (за 34 итерации):
\begin{align}
    D_1 =& (0.30991939238709043, 0.36480591771144494, 0.5777524438528959) \text{,}\nonumber\\
    D_2 =& (-0.6999267004284343, -0.19986747344005598, 0.09962177039906268) \text{.}\nonumber
\end{align}

Можно заметить, что в случае неполной сферы полученное приближение оказалось менее точным.
