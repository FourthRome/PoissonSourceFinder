\section{Численный метод решения обратных задач}

В качестве численного метода решения обратных задач 1 и 2 в вариационной постановке предлагается использовать градиентный спуск. Минимизировать предстоит функции $\Phi$ и $\Phi_{\Sigma_1}$ соответственно.

На практике удобнее проводить минимизацию не на множестве допустимых векторов $\vect{v}$, задающих декартовы координаты источников, а на множестве соответствующих им векторов из сферических координат $\vect{w}(\vect{v})$, $\rho_i < 1$, $\overline{1,N}$. В соответствии с указанными ранее преобразованиями координат считаем $\rho_i < 1$, $\phi_i \in [0; 2\pi)$, $\theta_i \in [0; \pi]$.

Для определения направления спуска нам потребуются частные производные функций $\Phi$ и $\Phi_{\Sigma_1}$ по сферическим координатам $i$-го источника, $i = \overline{1,N}$; вычислим их для $\Phi$.

Производная по $\rho_i$:
\begin{flalign}
	% Header
	\frac{\partial \Phi}{\partial \rho_i}
	=&
	\frac{\partial}{\partial \rho_i} \Phi(\vect{w}; Q) = 
	\frac{\partial}{\partial \rho_i} \Phi(\rho_1, \phi_1, \theta_1, ..., \rho_i, \phi_i, \theta_i, ..., \rho_n, \phi_n, \theta_n; Q)
	=&&\nonumber\\[30pt]
	% First partial
	=&
	\frac{\partial}{\partial \rho_i} 
	\bigg[
	\int\limits_{0}^{2\pi}\int\limits_{0}^{\pi}
	\big[g(\phi, \theta; \vect{w}) - Q(\phi,\theta)\big]^2 
	\sin{\theta} \, d\phi \, d\theta
	\bigg]
	=&&\nonumber\\[30pt]
	% External derivative 
	=&
	2
	\cdot
	\int\limits_{0}^{2\pi}\int\limits_{0}^{\pi}
	\big[g(\phi, \theta; \vect{w}) - Q(\phi,\theta)\big]
	\cdot
	\frac{\partial}{\partial \rho_i}
	g(\phi, \theta; \vect{w})
	\cdot
	\sin{\theta} \, d\phi \, d\theta
	=&&\nonumber\\[30pt]
	% Internal derivative
	=&
	2
	\cdot
	\int\limits_{0}^{2\pi}\int\limits_{0}^{\pi}
	\big[g(\phi, \theta; \vect{w}) - Q(\phi,\theta)\big]
	\cdot&&\nonumber\\[10pt]
	\cdot&
	\bigg[
	(-1)
	\cdot
	\frac{-2\rho_i K_i(\phi, \theta; \vect{w})^{\tfrac{3}{2}}
		-
		(1 - \rho_i^2)
		\cdot
		\tfrac{3}{2}
		K_i(\phi, \theta; \vect{w})^{\tfrac{1}{2}}
		\cdot
		\frac{\partial}{\partial \rho_i}
		K_i(\phi, \theta; \vect{w})
	}
	{K_i(\phi, \theta; \vect{w})^3}
	\bigg]
	\cdot
	\sin{\theta} \, d\phi \, d\theta
	=&&\nonumber\\[30pt]
	% Reduction
	=&
	\int\limits_{0}^{2\pi}\int\limits_{0}^{\pi}
	\big[g(\phi, \theta; \vect{w}) - Q(\phi,\theta)\big]
	\cdot&&\nonumber\\[10pt]
	\cdot&
	\frac{4 \rho_i K_i(\phi, \theta; \vect{w})
		+ 3 (1 - \rho_i^2)
		\big[
		2\rho_i - 2
		(\cos(\phi - \phi_i)\sin\theta\sin\theta_i + \cos\theta\cos\theta_i)
		\big]}
	{K_i(\phi, \theta; \vect{w})^{\tfrac{5}{2}}}
	\cdot
	\sin{\theta} \, d\phi \, d\theta
	=&&\nonumber\\[30pt]
	% Final
	=&
	\int\limits_{0}^{2\pi}\int\limits_{0}^{\pi}
	\big[
	g(\phi, \theta; \vect{w})
	- Q(\phi,\theta)
	\big]
	\cdot&&\nonumber\\[10pt]
	\cdot&
	\bigg[
	\frac{4\rho_i}{K_i(\phi, \theta; \vect{w})^{\tfrac{3}{2}}}
	+
	6 \cdot
	\frac{(1 - \rho_i^2)
		\big[
		\rho_i - (\cos(\phi - \phi_i)\sin\theta\sin\theta_i + \cos\theta\cos\theta_i)
		\big]}
	{K_i(\phi, \theta; \vect{w})^{\tfrac{5}{2}}}
	\bigg]
	\cdot&&\nonumber\\[10pt]
	\cdot&
	\sin{\theta} \, d\phi \, d\theta
	\text{.}&&
\end{flalign}

Производная по $\phi_i$:
\begin{flalign}
	% Header
	\frac{\partial \Phi}{\partial \phi_i}
	=&
	\frac{\partial}{\partial \phi_i} \Phi(\vect{w}; Q) = 
	\frac{\partial}{\partial \phi_i} \Phi(\rho_1, \phi_1, \theta_1, ..., \rho_i, \phi_i, \theta_i, ..., \rho_n, \phi_n, \theta_n; Q)
	=&&\nonumber\\[30pt]
	% First partial
	=&
	\frac{\partial}{\partial \phi_i} 
	\bigg[
	\int\limits_{0}^{2\pi}\int\limits_{0}^{\pi}
	\big[g(\phi, \theta; \vect{w}) - Q(\phi,\theta)\big]^2 
	\sin{\theta} \, d\phi \, d\theta
	\bigg]
	=&&\nonumber\\[30pt]
	% External derivative 
	=&
	2 \cdot
	\int\limits_{0}^{2\pi}\int\limits_{0}^{\pi}
	\big[g(\phi, \theta; \vect{w}) - Q(\phi,\theta)\big]
	\cdot
	\frac{\partial}{\partial \phi_i}
	g(\phi, \theta; \vect{w})
	\sin{\theta} \, d\phi \, d\theta
	=&&\nonumber\\[30pt]
	% Internal derivative
	=&
	2 \cdot
	\int\limits_{0}^{2\pi}\int\limits_{0}^{\pi}
	\big[g(\phi, \theta; \vect{w}) - Q(\phi,\theta)\big]
	\cdot&&\nonumber\\[10pt]
	\cdot&
	\bigg[
	(-1)
	\cdot
	\frac{
		-
		(1 - \rho_i^2)
		\cdot
		\tfrac{3}{2}
		K_i(\phi, \theta; \vect{w})^{\tfrac{1}{2}}
		\cdot
		\frac{\partial}{\partial \phi_i}
		K_i(\phi, \theta; \vect{w})
	}
	{K_i(\phi, \theta; \vect{w})^3}
	\bigg]
	\cdot
	\sin{\theta} \, d\phi \, d\theta
	=&&\nonumber\\[30pt]
	% Reduction
	=&
	\int\limits_{0}^{2\pi}\int\limits_{0}^{\pi}
	\big[g(\phi, \theta; \vect{w}) - Q(\phi,\theta)\big]
	\cdot&&\nonumber\\[10pt]
	\cdot&
	\frac{3 (1 - \rho_i^2)( - 2 \rho_i
		\sin(\phi - \phi_i)\sin\theta\sin\theta_i
		)}
	{K_i(\phi, \theta; \vect{w})^{\tfrac{5}{2}}}
	\cdot
	\sin{\theta} \, d\phi \, d\theta
	=&&\nonumber\\[30pt]
	% Final
	=&
	-6
	\int\limits_{0}^{2\pi}\int\limits_{0}^{\pi}
	\big[
	g(\phi, \theta; \vect{w})
	- Q(\phi,\theta)
	\big]
	\cdot
	\bigg[
	\frac{(1 - \rho_i^2)\rho_i\sin(\phi - \phi_i)\sin{\theta}\sin{\theta_i}}
	{K_i(\phi, \theta; \vect{w})^{\tfrac{5}{2}}}
	\bigg]
	\cdot
	\sin{\theta} \, d\phi \, d\theta
	\text{.}&&
\end{flalign}

Производная по $\theta_i$:
\begin{flalign}
	% Header
	\frac{\partial \Phi}{\partial \theta_i}
	=&
	\frac{\partial}{\partial \theta_i} \Phi(\vect{w}; Q) = 
	\frac{\partial}{\partial \theta_i} \Phi(\rho_1, \phi_1, \theta_1, ..., \rho_i, \phi_i, \theta_i, ..., \rho_n, \phi_n, \theta_n; Q)
	=&&\nonumber\\[30pt]
	% First partial
	=&
	\frac{\partial}{\partial \theta_i} 
	\bigg[
	\int\limits_{0}^{2\pi}\int\limits_{0}^{\pi}
	\big[g(\phi, \theta; \vect{w}) - Q(\phi,\theta)\big]^2 
	\sin\theta \, d\phi \, d\theta
	\bigg]
	=&&\nonumber\\[30pt]
	% External derivative 
	=&
	2 \cdot
	\int\limits_{0}^{2\pi}\int\limits_{0}^{\pi}
	\big[g(\phi, \theta; \vect{w}) - Q(\phi,\theta)\big]
	\cdot
	\frac{\partial}{\partial \theta_i}
	g(\phi, \theta; \vect{w})
	\sin{\theta} \, d\phi \, d\theta
	=&&\nonumber\\[30pt]
	% Internal derivative
	=&
	2 \cdot
	\int\limits_{0}^{2\pi}\int\limits_{0}^{\pi}
	\big[g(\phi, \theta; \vect{w}) - Q(\phi,\theta)\big]
	\cdot&&\nonumber\\[10pt]
	\cdot&
	\bigg[
	(-1)
	\cdot
	\frac{
		-
		(1 - \rho_i^2)
		\cdot
		\tfrac{3}{2}
		K_i(\phi, \theta; \vect{w})^{\tfrac{1}{2}}
		\cdot
		\frac{\partial}{\partial \theta_i}
		K_i(\phi, \theta; \vect{w})
	}
	{K_i(\phi, \theta; \vect{w})^3}
	\bigg]
	\cdot
	\sin\theta \, d\phi \, d\theta
	=&&\nonumber\\[30pt]
	% Reduction
	=&
	\int\limits_{0}^{2\pi}\int\limits_{0}^{\pi}
	\big[g(\phi, \theta; \vect{w}) - Q(\phi,\theta)\big]
	\cdot&&\nonumber\\[10pt]
	\cdot&
	\frac{3 (1 - \rho_i^2)
		\big[
		-2\rho_i
		(\cos(\phi - \phi_i)\sin\theta\cos\theta_i
		-
		\cos\theta\sin\theta_i)
		\big]}
	{K_i(\phi, \theta; \vect{w})^{\tfrac{5}{2}}}
	\cdot
	\sin{\theta} \, d\phi \, d\theta
	=&&\nonumber\\[30pt]
	% Final
	=&
	-6
	\int\limits_{0}^{2\pi}\int\limits_{0}^{\pi}
	\big[
	g(\phi, \theta; \vect{w})
	- Q(\phi,\theta)
	\big]
	\cdot&&\nonumber\\[10pt]
	\cdot&
	\bigg[
	\frac{(1 - \rho_i^2)
		\rho_i(\cos(\phi - \phi_i)\sin\theta\cos\theta_i - \cos\theta\sin\theta_i)}
	{K_i(\phi, \theta; \vect{w})^{\tfrac{5}{2}}}
	\bigg]
	\cdot
	\sin{\theta} \, d\phi \, d\theta
	\text{.}&&
\end{flalign}

\newpage
\textbf{Итоговые частные производные функции $\Phi$:}
\begin{flalign}
	% Header
	\frac{\partial \Phi}{\partial \rho_i}
	=&
	\int\limits_{0}^{2\pi}\int\limits_{0}^{\pi}
	\big[
	g(\phi, \theta; \vect{w})
	- Q(\phi,\theta)
	\big] \cdot
	\bigg[
	\frac{4\rho_i}{K_i(\phi, \theta; \vect{w})^{\tfrac{3}{2}}}
	+&&\nonumber\\[10pt]
	+&
	6 \cdot
	\frac{(1 - \rho_i^2)
		\big[
		\rho_i - (\cos(\phi - \phi_i)\sin\theta\sin\theta_i + \cos\theta\cos\theta_i)
		\big]}
	{K_i(\phi, \theta; \vect{w})^{\tfrac{5}{2}}}
	\bigg]
	\cdot
	\sin{\theta} \, d\phi \, d\theta
	\text{,}&&
\end{flalign}
\begin{flalign}
	% Header
	\frac{\partial \Phi}{\partial \phi_i}
	=&
	-6
	\int\limits_{0}^{2\pi}\int\limits_{0}^{\pi}
	\big[
	g(\phi, \theta; \vect{w})
	- Q(\phi,\theta)
	\big]
	\cdot&&\nonumber\\[10pt]
	\cdot&
	\bigg[
	\frac{(1 - \rho_i^2)\rho_i\sin(\phi - \phi_i)\sin{\theta}\sin{\theta_i}}
	{K_i(\phi, \theta; \vect{w})^{\tfrac{5}{2}}}
	\bigg]
	\cdot
	\sin{\theta} \, d\phi \, d\theta
	\text{,}&&
\end{flalign}
\begin{flalign}
	% Header
	\frac{\partial \Phi}{\partial \theta_i}
	=&
	-6
	\int\limits_{0}^{2\pi}\int\limits_{0}^{\pi}
	\big[
	g(\phi, \theta; \vect{w})
	- Q(\phi,\theta)
	\big]
	\cdot&&\nonumber\\[10pt]
	\cdot&
	\bigg[
	\frac{(1 - \rho_i^2)
		\rho_i(\cos(\phi - \phi_i)\sin\theta\cos\theta_i - \cos\theta\sin\theta_i)}
	{K_i(\phi, \theta; \vect{w})^{\tfrac{5}{2}}}
	\bigg]
	\cdot
	\sin{\theta} \, d\phi \, d\theta
	\text{.}&&
\end{flalign}

Частные производные функции $\Phi_{\Sigma_1}$ вычисляются аналогично, поэтому приведём лишь итоговые формулы:
\begin{flalign}
	% Header
	\frac{\partial \Phi_{\Sigma_1}}{\partial \rho_i}
	=&
	\int\limits_{\phi_1}^{\phi_2}\int\limits_{\theta_1}^{\theta_2}
	\big[
	g(\phi, \theta; \vect{w})
	- Q(\phi,\theta)
	\big] \cdot
	\bigg[
	\frac{4\rho_i}{K_i(\phi, \theta; \vect{w})^{\tfrac{3}{2}}}
	+&&\nonumber\\[10pt]
	+&
	6 \cdot
	\frac{(1 - \rho_i^2)
		\big[
		\rho_i - (\cos(\phi - \phi_i)\sin\theta\sin\theta_i + \cos\theta\cos\theta_i)
		\big]}
	{K_i(\phi, \theta; \vect{w})^{\tfrac{5}{2}}}
	\bigg]
	\cdot
	\sin{\theta} \, d\phi \, d\theta
	\text{,}&&
\end{flalign}
\begin{flalign}
	% Header
	\frac{\partial \Phi_{\Sigma_1}}{\partial \phi_i}
	=&
	-6
	\int\limits_{\phi_1}^{\phi_2}\int\limits_{\theta_1}^{\theta_2}
	\big[
	g(\phi, \theta; \vect{w})
	- Q(\phi,\theta)
	\big]
	\cdot&&\nonumber\\[10pt]
	\cdot&
	\bigg[
	\frac{(1 - \rho_i^2)\rho_i\sin(\phi - \phi_i)\sin{\theta}\sin{\theta_i}}
	{K_i(\phi, \theta; \vect{w})^{\tfrac{5}{2}}}
	\bigg]
	\cdot
	\sin{\theta} \, d\phi \, d\theta
	\text{,}&&
\end{flalign}
\begin{flalign}
	% Header
	\frac{\partial \Phi_{\Sigma_1}}{\partial \theta_i}
	=&
	-6
	\int\limits_{\phi_1}^{\phi_2}\int\limits_{\theta_1}^{\theta_2}
	\big[
	g(\phi, \theta; \vect{w})
	- Q(\phi,\theta)
	\big]
	\cdot&&\nonumber\\[10pt]
	\cdot&
	\bigg[
	\frac{(1 - \rho_i^2)
		\rho_i(\cos(\phi - \phi_i)\sin\theta\cos\theta_i - \cos\theta\sin\theta_i)}
	{K_i(\phi, \theta; \vect{w})^{\tfrac{5}{2}}}
	\bigg]
	\cdot
	\sin{\theta} \, d\phi \, d\theta
	\text{.}&&
\end{flalign}

Опишем применение градиентного метода для решения задачи минимизации функции.



Градиентный спуск будем осуществлять следующим образом: вычислив для очередного приближения координат $\vect{v}_k$, $k=1,2,...$ градиент $\nabla \Phi$ ($\nabla \Phi_{\Sigma_1}$), будем дробить по степеням двойки шаг в противоположном направлении до тех пор, пока получаемое приближение $\vect{v}_{k+1}$ не даст меньшее значение функции $\Phi$ ($\Phi_{\Sigma_1}$), чем текущее приближение.

Данный процесс повторяется, пока функция $\Phi$ ($\Phi_{\Sigma_1}$) не примет значение ниже порогового либо пока изменение координат $\Delta \vect{v}_k = \vect{v}_{k+1} - \vect{v}_k$ не станет слишком мало по норме.

\[
\vect{w}_k = \vect{w}_{k - 1} - \alpha_k \nabla \Phi(\vect{w}; Q)
\]

\[
\left\lVert \vect{w}_k - \vect{w}_{k - 1} \right\lVert < \varepsilon
\]