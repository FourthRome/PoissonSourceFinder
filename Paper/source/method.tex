\section{Численный метод решения обратных задач}

В качестве численного метода решения обратных задач 1 и 2 в вариационной постановке предлагается использовать градиентный спуск. Минимизировать предстоит функции $\Phi$ и $\Phi_{\Sigma_1}$ соответственно.

На практике удобнее проводить минимизацию не на множестве допустимых векторов $\vect{v}$, задающих декартовы координаты источников, а на множестве соответствующих им векторов из сферических координат $\vect{w}(\vect{v})$, $\rho_i < 1$, $\overline{1,N}$. В соответствии с указанными ранее преобразованиями координат считаем $\rho_i < 1$, $\phi_i \in [0; 2\pi)$, $\theta_i \in [0; \pi]$.

Для определения направления спуска нам потребуются частные производные функций $\Phi$ и $\Phi_{\Sigma_1}$ по сферическим координатам $i$-го источника, $i = \overline{1,N}$; вычислим их для $\Phi$.

Производная по $\rho_i$:
\begin{flalign}
	% Header
	\frac{\partial \Phi}{\partial \rho_i}
	=&
	\frac{\partial}{\partial \rho_i} \Phi(\vect{w}; Q) = 
	\frac{\partial}{\partial \rho_i} \Phi(\rho_1, \phi_1, \theta_1, ..., \rho_i, \phi_i, \theta_i, ..., \rho_n, \phi_n, \theta_n; Q)
	=&&\nonumber\\[30pt]
	% First partial
	=&
	\frac{\partial}{\partial \rho_i} 
	\bigg[
	\int\limits_{0}^{2\pi}\int\limits_{0}^{\pi}
	\big[g(\phi, \theta; \vect{w}) - Q(\phi,\theta)\big]^2 
	\sin{\theta} \, d\phi \, d\theta
	\bigg]
	=&&\nonumber\\[30pt]
	% External derivative 
	=&
	2
	\cdot
	\int\limits_{0}^{2\pi}\int\limits_{0}^{\pi}
	\big[g(\phi, \theta; \vect{w}) - Q(\phi,\theta)\big]
	\cdot
	\frac{\partial}{\partial \rho_i}
	g(\phi, \theta; \vect{w})
	\cdot
	\sin{\theta} \, d\phi \, d\theta
	=&&\nonumber\\[30pt]
	% Internal derivative
	=&
	2
	\cdot
	\int\limits_{0}^{2\pi}\int\limits_{0}^{\pi}
	\big[g(\phi, \theta; \vect{w}) - Q(\phi,\theta)\big]
	\cdot&&\nonumber\\[10pt]
	\cdot&
	\bigg[
	(-1)
	\cdot
	\frac{-2\rho_i K_i(\phi, \theta; \vect{w})^{\tfrac{3}{2}}
		-
		(1 - \rho_i^2)
		\cdot
		\tfrac{3}{2}
		K_i(\phi, \theta; \vect{w})^{\tfrac{1}{2}}
		\cdot
		\frac{\partial}{\partial \rho_i}
		K_i(\phi, \theta; \vect{w})
	}
	{K_i(\phi, \theta; \vect{w})^3}
	\bigg]
	\cdot
	\sin{\theta} \, d\phi \, d\theta
	=&&\nonumber\\[30pt]
	% Reduction
	=&
	\int\limits_{0}^{2\pi}\int\limits_{0}^{\pi}
	\big[g(\phi, \theta; \vect{w}) - Q(\phi,\theta)\big]
	\cdot&&\nonumber\\[10pt]
	\cdot&
	\frac{4 \rho_i K_i(\phi, \theta; \vect{w})
		+ 3 (1 - \rho_i^2)
		\big[
		2\rho_i - 2
		(\cos(\phi - \phi_i)\sin\theta\sin\theta_i + \cos\theta\cos\theta_i)
		\big]}
	{K_i(\phi, \theta; \vect{w})^{\tfrac{5}{2}}}
	\cdot
	\sin{\theta} \, d\phi \, d\theta
	=&&\nonumber\\[30pt]
	% Final
	=&
	\int\limits_{0}^{2\pi}\int\limits_{0}^{\pi}
	\big[
	g(\phi, \theta; \vect{w})
	- Q(\phi,\theta)
	\big]
	\cdot&&\nonumber\\[10pt]
	\cdot&
	\bigg[
	\frac{4\rho_i}{K_i(\phi, \theta; \vect{w})^{\tfrac{3}{2}}}
	+
	6 \cdot
	\frac{(1 - \rho_i^2)
		\big[
		\rho_i - (\cos(\phi - \phi_i)\sin\theta\sin\theta_i + \cos\theta\cos\theta_i)
		\big]}
	{K_i(\phi, \theta; \vect{w})^{\tfrac{5}{2}}}
	\bigg]
	\cdot&&\nonumber\\[10pt]
	\cdot&
	\sin{\theta} \, d\phi \, d\theta
	\text{.}&&
\end{flalign}

Производная по $\phi_i$:
\begin{flalign}
	% Header
	\frac{\partial \Phi}{\partial \phi_i}
	=&
	\frac{\partial}{\partial \phi_i} \Phi(\vect{w}; Q) = 
	\frac{\partial}{\partial \phi_i} \Phi(\rho_1, \phi_1, \theta_1, ..., \rho_i, \phi_i, \theta_i, ..., \rho_n, \phi_n, \theta_n; Q)
	=&&\nonumber\\[30pt]
	% First partial
	=&
	\frac{\partial}{\partial \phi_i} 
	\bigg[
	\int\limits_{0}^{2\pi}\int\limits_{0}^{\pi}
	\big[g(\phi, \theta; \vect{w}) - Q(\phi,\theta)\big]^2 
	\sin{\theta} \, d\phi \, d\theta
	\bigg]
	=&&\nonumber\\[30pt]
	% External derivative 
	=&
	2 \cdot
	\int\limits_{0}^{2\pi}\int\limits_{0}^{\pi}
	\big[g(\phi, \theta; \vect{w}) - Q(\phi,\theta)\big]
	\cdot
	\frac{\partial}{\partial \phi_i}
	g(\phi, \theta; \vect{w})
	\sin{\theta} \, d\phi \, d\theta
	=&&\nonumber\\[30pt]
	% Internal derivative
	=&
	2 \cdot
	\int\limits_{0}^{2\pi}\int\limits_{0}^{\pi}
	\big[g(\phi, \theta; \vect{w}) - Q(\phi,\theta)\big]
	\cdot&&\nonumber\\[10pt]
	\cdot&
	\bigg[
	(-1)
	\cdot
	\frac{
		-
		(1 - \rho_i^2)
		\cdot
		\tfrac{3}{2}
		K_i(\phi, \theta; \vect{w})^{\tfrac{1}{2}}
		\cdot
		\frac{\partial}{\partial \phi_i}
		K_i(\phi, \theta; \vect{w})
	}
	{K_i(\phi, \theta; \vect{w})^3}
	\bigg]
	\cdot
	\sin{\theta} \, d\phi \, d\theta
	=&&\nonumber\\[30pt]
	% Reduction
	=&
	\int\limits_{0}^{2\pi}\int\limits_{0}^{\pi}
	\big[g(\phi, \theta; \vect{w}) - Q(\phi,\theta)\big]
	\cdot&&\nonumber\\[10pt]
	\cdot&
	\frac{3 (1 - \rho_i^2)( - 2 \rho_i
		\sin(\phi - \phi_i)\sin\theta\sin\theta_i
		)}
	{K_i(\phi, \theta; \vect{w})^{\tfrac{5}{2}}}
	\cdot
	\sin{\theta} \, d\phi \, d\theta
	=&&\nonumber\\[30pt]
	% Final
	=&
	-6
	\int\limits_{0}^{2\pi}\int\limits_{0}^{\pi}
	\big[
	g(\phi, \theta; \vect{w})
	- Q(\phi,\theta)
	\big]
	\cdot
	\bigg[
	\frac{(1 - \rho_i^2)\rho_i\sin(\phi - \phi_i)\sin{\theta}\sin{\theta_i}}
	{K_i(\phi, \theta; \vect{w})^{\tfrac{5}{2}}}
	\bigg]
	\cdot
	\sin{\theta} \, d\phi \, d\theta
	\text{.}&&
\end{flalign}

Производная по $\theta_i$:
\begin{flalign}
	% Header
	\frac{\partial \Phi}{\partial \theta_i}
	=&
	\frac{\partial}{\partial \theta_i} \Phi(\vect{w}; Q) = 
	\frac{\partial}{\partial \theta_i} \Phi(\rho_1, \phi_1, \theta_1, ..., \rho_i, \phi_i, \theta_i, ..., \rho_n, \phi_n, \theta_n; Q)
	=&&\nonumber\\[30pt]
	% First partial
	=&
	\frac{\partial}{\partial \theta_i} 
	\bigg[
	\int\limits_{0}^{2\pi}\int\limits_{0}^{\pi}
	\big[g(\phi, \theta; \vect{w}) - Q(\phi,\theta)\big]^2 
	\sin\theta \, d\phi \, d\theta
	\bigg]
	=&&\nonumber\\[30pt]
	% External derivative 
	=&
	2 \cdot
	\int\limits_{0}^{2\pi}\int\limits_{0}^{\pi}
	\big[g(\phi, \theta; \vect{w}) - Q(\phi,\theta)\big]
	\cdot
	\frac{\partial}{\partial \theta_i}
	g(\phi, \theta; \vect{w})
	\sin{\theta} \, d\phi \, d\theta
	=&&\nonumber\\[30pt]
	% Internal derivative
	=&
	2 \cdot
	\int\limits_{0}^{2\pi}\int\limits_{0}^{\pi}
	\big[g(\phi, \theta; \vect{w}) - Q(\phi,\theta)\big]
	\cdot&&\nonumber\\[10pt]
	\cdot&
	\bigg[
	(-1)
	\cdot
	\frac{
		-
		(1 - \rho_i^2)
		\cdot
		\tfrac{3}{2}
		K_i(\phi, \theta; \vect{w})^{\tfrac{1}{2}}
		\cdot
		\frac{\partial}{\partial \theta_i}
		K_i(\phi, \theta; \vect{w})
	}
	{K_i(\phi, \theta; \vect{w})^3}
	\bigg]
	\cdot
	\sin\theta \, d\phi \, d\theta
	=&&\nonumber\\[30pt]
	% Reduction
	=&
	\int\limits_{0}^{2\pi}\int\limits_{0}^{\pi}
	\big[g(\phi, \theta; \vect{w}) - Q(\phi,\theta)\big]
	\cdot&&\nonumber\\[10pt]
	\cdot&
	\frac{3 (1 - \rho_i^2)
		\big[
		-2\rho_i
		(\cos(\phi - \phi_i)\sin\theta\cos\theta_i
		-
		\cos\theta\sin\theta_i)
		\big]}
	{K_i(\phi, \theta; \vect{w})^{\tfrac{5}{2}}}
	\cdot
	\sin{\theta} \, d\phi \, d\theta
	=&&\nonumber\\[30pt]
	% Final
	=&
	-6
	\int\limits_{0}^{2\pi}\int\limits_{0}^{\pi}
	\big[
	g(\phi, \theta; \vect{w})
	- Q(\phi,\theta)
	\big]
	\cdot&&\nonumber\\[10pt]
	\cdot&
	\bigg[
	\frac{(1 - \rho_i^2)
		\rho_i(\cos(\phi - \phi_i)\sin\theta\cos\theta_i - \cos\theta\sin\theta_i)}
	{K_i(\phi, \theta; \vect{w})^{\tfrac{5}{2}}}
	\bigg]
	\cdot
	\sin{\theta} \, d\phi \, d\theta
	\text{.}&&
\end{flalign}

\newpage
\textbf{Итоговые частные производные функции $\Phi$:}
\begin{flalign}
	% Header
	\frac{\partial \Phi}{\partial \rho_i}
	=&
	\int\limits_{0}^{2\pi}\int\limits_{0}^{\pi}
	\big[
	g(\phi, \theta; \vect{w})
	- Q(\phi,\theta)
	\big] \cdot
	\bigg[
	\frac{4\rho_i}{K_i(\phi, \theta; \vect{w})^{\tfrac{3}{2}}}
	+&&\nonumber\\[10pt]
	+&
	6 \cdot
	\frac{(1 - \rho_i^2)
		\big[
		\rho_i - (\cos(\phi - \phi_i)\sin\theta\sin\theta_i + \cos\theta\cos\theta_i)
		\big]}
	{K_i(\phi, \theta; \vect{w})^{\tfrac{5}{2}}}
	\bigg]
	\cdot
	\sin{\theta} \, d\phi \, d\theta
	\text{,}&&
\end{flalign}
\begin{flalign}
	% Header
	\frac{\partial \Phi}{\partial \phi_i}
	=&
	-6
	\int\limits_{0}^{2\pi}\int\limits_{0}^{\pi}
	\big[
	g(\phi, \theta; \vect{w})
	- Q(\phi,\theta)
	\big]
	\cdot&&\nonumber\\[10pt]
	\cdot&
	\bigg[
	\frac{(1 - \rho_i^2)\rho_i\sin(\phi - \phi_i)\sin{\theta}\sin{\theta_i}}
	{K_i(\phi, \theta; \vect{w})^{\tfrac{5}{2}}}
	\bigg]
	\cdot
	\sin{\theta} \, d\phi \, d\theta
	\text{,}&&
\end{flalign}
\begin{flalign}
	% Header
	\frac{\partial \Phi}{\partial \theta_i}
	=&
	-6
	\int\limits_{0}^{2\pi}\int\limits_{0}^{\pi}
	\big[
	g(\phi, \theta; \vect{w})
	- Q(\phi,\theta)
	\big]
	\cdot&&\nonumber\\[10pt]
	\cdot&
	\bigg[
	\frac{(1 - \rho_i^2)
		\rho_i(\cos(\phi - \phi_i)\sin\theta\cos\theta_i - \cos\theta\sin\theta_i)}
	{K_i(\phi, \theta; \vect{w})^{\tfrac{5}{2}}}
	\bigg]
	\cdot
	\sin{\theta} \, d\phi \, d\theta
	\text{.}&&
\end{flalign}

Частные производные функции $\Phi_{\Sigma_1}$ вычисляются аналогично, поэтому приведём лишь итоговые формулы:
\begin{flalign}
	% Header
	\frac{\partial \Phi_{\Sigma_1}}{\partial \rho_i}
	=&
	\int\limits_{\phi_1}^{\phi_2}\int\limits_{\theta_1}^{\theta_2}
	\big[
	g(\phi, \theta; \vect{w})
	- Q(\phi,\theta)
	\big] \cdot
	\bigg[
	\frac{4\rho_i}{K_i(\phi, \theta; \vect{w})^{\tfrac{3}{2}}}
	+&&\nonumber\\[10pt]
	+&
	6 \cdot
	\frac{(1 - \rho_i^2)
		\big[
		\rho_i - (\cos(\phi - \phi_i)\sin\theta\sin\theta_i + \cos\theta\cos\theta_i)
		\big]}
	{K_i(\phi, \theta; \vect{w})^{\tfrac{5}{2}}}
	\bigg]
	\cdot
	\sin{\theta} \, d\phi \, d\theta
	\text{,}&&
\end{flalign}
\begin{flalign}
	% Header
	\frac{\partial \Phi_{\Sigma_1}}{\partial \phi_i}
	=&
	-6
	\int\limits_{\phi_1}^{\phi_2}\int\limits_{\theta_1}^{\theta_2}
	\big[
	g(\phi, \theta; \vect{w})
	- Q(\phi,\theta)
	\big]
	\cdot&&\nonumber\\[10pt]
	\cdot&
	\bigg[
	\frac{(1 - \rho_i^2)\rho_i\sin(\phi - \phi_i)\sin{\theta}\sin{\theta_i}}
	{K_i(\phi, \theta; \vect{w})^{\tfrac{5}{2}}}
	\bigg]
	\cdot
	\sin{\theta} \, d\phi \, d\theta
	\text{,}&&
\end{flalign}
\newpage
\begin{flalign}
	% Header
	\frac{\partial \Phi_{\Sigma_1}}{\partial \theta_i}
	=&
	-6
	\int\limits_{\phi_1}^{\phi_2}\int\limits_{\theta_1}^{\theta_2}
	\big[
	g(\phi, \theta; \vect{w})
	- Q(\phi,\theta)
	\big]
	\cdot&&\nonumber\\[10pt]
	\cdot&
	\bigg[
	\frac{(1 - \rho_i^2)
		\rho_i(\cos(\phi - \phi_i)\sin\theta\cos\theta_i - \cos\theta\sin\theta_i)}
	{K_i(\phi, \theta; \vect{w})^{\tfrac{5}{2}}}
	\bigg]
	\cdot
	\sin{\theta} \, d\phi \, d\theta
	\text{.}&&
\end{flalign}

\emph{Опишем градиентный метод} решения задачи минимизации функции. За основу берётся метод из \cite{vasil'yev} с небольшими изменениями. Изложение будем вести для функции $\Phi$, для $\Phi_{\Sigma_1}$ рассуждения аналогичны приведённым ниже.

В качестве начального приближения для небольшого $N$ ($N \le 7$) будем рассматривать подмножества без повторов из $N$ элементов следующего множества точек (заданных в декартовых координатах):
\[
\lbrace
(0, 0, 0)\text{, }
(0.5, 0, 0)\text{, }
(-0.5, 0, 0)\text{, }
(0, 0.5, 0)\text{, }
(0, -0.5, 0)\text{, }
(0, 0, 0.5)\text{, }
(0, 0, -0.5)
\rbrace
\text{.}
\]

Путём перебора всевозможных сочетаний по $N$ точек из приведённых семи выберем начальное приближение, на котором значение функции $\Phi$ минимально. Координаты этих точек образуют вектор $\vect{v}_1$. Предложенный метод позволяет найти близкое начальное приближение и в то же время помогает избежать "слипания" источников друг с другом на начальном этапе.

Градиентный спуск будем осуществлять следующим образом: вычислим для очередного (с номером $k$) приближения координат $\vect{v}_k$, $k=1,2,...$ градиент $\nabla \Phi(\vect{w}(\vect{v}_k))$. Получим $N + 1$ возможных направлений движения:
\begin{align}
    \vect{s}_{k0} =& \nabla \Phi(\vect{w}(\vect{v}_k))
    =
    \nabla \Phi(\vect{w}_k)
    \text{,}\\
    \vect{s}_{ki} =& (0,0,0,...,
    \frac{\partial}{\partial \rho_i} \Phi(\vect{w}_k),
    \frac{\partial}{\partial \phi_i} \Phi(\vect{w}_k),
    \frac{\partial}{\partial \theta_i} \Phi(\vect{w}_k),
    ...,0,0,0)
    \text{, }
    i = \overline{1,N}
    \text{.}
\end{align}

Начинаем с величины шага $\alpha_k^1 = 1$. Выбираем направление, в котором следует делать шаг данной величины:
\begin{equation}
    \vect{s}_k^1 = \argmin_{\vect{s}_{ki}} \Phi(\vect{w}_k - \alpha_k^1 \vect{s}_{ki}) \text{.}
\end{equation}

Далее для полученного направления $\vect{s}_k^1$ проверяется, выполнено ли
\begin{align}
    &\vect{w}_k - \alpha_k^1 \vect{s}_k^1 \in T \text{,}\\
    &\Phi(\vect{w}_k - \alpha_k^1 \vect{s}_k^1) < \Phi(\vect{w}_k) \text{.}
\end{align}
Если это условие выполнено, то полагаем $\vect{w}_{k+1} = \vect{w}_{k} - \alpha_k^1 \vect{s}_k^1$ и переходим к $k + 1$ итерации. В противном случае уменьшаем величину шага: $\alpha_k^{2} = \frac{\alpha_k^1}{2}$ или, в общем случае,
\begin{equation}
    \alpha_k^{j} = \frac{\alpha_k^{j - 1}}{2} \text{, } j = 2,3,... \text{,}
\end{equation}
 и повторяем описанный выше процесс выбора направления и его проверки:
\begin{align}
    &\vect{s}_k^j = \argmin_{\vect{s}_{ki}} \Phi(\vect{w}_k - \alpha_k^j \vect{s}_{ki}) \text{,}\\
    &\vect{w}_k - \alpha_k^j \vect{s}_k^j \in T \text{,}\\
    &\Phi(\vect{w}_k - \alpha_k^j \vect{s}_k^j) < \Phi(\vect{w}_k) \text{.}
\end{align}

Когда подходящие $\vect{s}_k^j$ и $\alpha_k^j$ найдены (для достаточно малой окрестности точки $\vect{v}_k$ это гарантировано в силу дифференцируемости), полагаем $\vect{s}_k = \vect{s}_k^j$ и $\alpha_k = \alpha_k^j$ и
\begin{equation}
\vect{w}_{k+1} = \vect{w}_{k} - \alpha_k \vect{s}_k \text{, } k=1,2,... \text{.}
\end{equation}

Данный процесс повторяется, пока изменение координат $\Delta \vect{v}_k = \vect{v}_{k+1} - \vect{v}_k$ не станет слишком мало по евклидовой норме пространства $\mathbb{R}^{3N}$:
\begin{equation}
\left\lVert \vect{v}_{k+1} - \vect{v}_{k} \right\lVert < \varepsilon \text{.}
\end{equation}

Описанный усложнённый по сравнению с движением в противоположном градиенту направлении метод возник в связи с неравным вкладом источников в общее значение $\Phi(\vect{w}(\vect{v}))$; как правило, чем ближе источник к поверхности, тем сильнее изменение его положения влияет на значение $\Phi$, что тормозит уточнение координат источников в случае, когда наиболее удалённый от центра сферы источник уже достаточно точно найден.

В приведённых выше формулах для $\Phi$, $\Phi_{\Sigma_1}$ и компонент градиента необходимо вычислять повторные интегралы по двум переменным. Делается это по равномерной (в фазовой плоскости) прямоугольной сетке с шагами $h_\phi = h_\theta = 0.01$ методом средних прямоугольников.

Для элемента поверхности
\begin{equation}
    \Delta \Sigma_{ij} = \lbrace (\rho, \phi, \theta)
    \text{ | }
    \rho = 1,\,
    i \cdot h_\phi \le \phi \le (i + 1) \cdot h_\phi,\,
    j \cdot h_\theta \le \theta \le (j + 1) \cdot h_\theta
    \rbrace
    \text{, }
    i,j=0,1,...\text{,}
\end{equation}
значение вычисляемой функции берётся в центральной точке
\begin{equation}
    P_{ij} = (\rho, \phi_i, \theta_j) = (1, (i + 0.5) \cdot h_\phi, (j + 0.5) \cdot h_\theta) \text{,}
\end{equation}
площадь равна
\begin{equation}
    S_{ij} = h_\phi \cdot (\cos j h_\theta - \cos\, (j + 1) h_\theta) \text{,}
\end{equation}
и потому повторный интеграл от произвольной функции $u(\phi, \theta)$ (где $u$ - это интегрируемая функция в $\Phi$, $\Phi_{\Sigma_1}$ или одной из частных производных этих функций) считается как
\begin{equation}
    \sum_{i, j}
    \big[
    u(\phi_i, \theta_i) \cdot S_{ij} \text{.}
    \big]
    \text{,}
\end{equation}
где сумма берётся по всем элементам $\Delta \Sigma_{ij}$, целиком лежащим в соответствующих частях поверхности сферы.

Предложенный метод был реализован в виде программного продукта на языке C\# (версия .NET 5.0), состоящего из библиотеки для проведения вычислений и консольного приложения для задания параметров и запуска вычислительных экспериментов. Для ускорения вычисления интегралов используется Task Parallel Library.

Библиотека позволяет задать различные параметры модели: число источников, которые предстоит искать, начальные приближения, шаги сетки интегрирования $h_\phi$ и $h_\theta$, часть поверхности, по которой должны вычисляться интегралы (под неё подстраивается сетка), желаемый критерий остановки и величину порогового значения для этого критерия и др.

Исходный код программной реализации доступен на GitHub \footnote{https://github.com/FourthRome/PoissonSourceFinder}.