\newpage
\section{Введение}

Математическое моделирование различных процессов и явлений играет большую роль в современной науке и технике. Оно находит применение как в фундаментальных исследованиях в физике, химии, биологии и других естественных науках, так и в прикладных задачах промышленности, медицины, экономики и т.д. \cite{samarskiy_models}. Возникающие при этом задачи можно разделить на два типа.

В первом случае известны внутренние параметры некой модели, и требуется на основе этих данных сделать количественные и качественные выводы о следующих из них явлениях. Например, по известным действующим на движущуюся материальную точку силам и её массе определить траекторию её движения. В таких задачах известна причина, и требуется определить следствие; их принято называть прямыми. Нередко подобные задачи встречаются в прикладном применении полученных ранее иными способами фундаментальных знаний в предметной области. Во втором случае, напротив, требуется по наблюдаемым внешним проявлениям определить внутренние параметры модели; по имеющемуся следствию установить причину. Задачи такого типа весьма характерны для научных исследований и называются обратными.

Обратные задачи часто связаны с обработкой экспериментальных данных. Это означает, что проблемы, вызванные принципиальными ограничениями проводимых экспериментов и доступными исследователям средствами, оказывают влияние и на принципы рассмотрения обратных задач. В связи с этим большое внимание в исследовании обратных задач уделяется понятию корректности постановки задачи, введённому французским математиком Жаком Адамаром \cite{hadamard}. Дадим определение корректности постановки задачи, следуя \cite{arsenin}.

Пусть $u$ - известные нам данные, $z$ - неизвестная информация, которую предстоит определить (т.е. решение задачи). Предположим, что $u \in U$ и $z \in Z$, где $U$ и $Z$ - некоторые метрические пространства. Тогда формально соответствие между этими данными можно записать в виде
\[
z = R(u) \text{, } u \in U \text{, } z \in Z \text{, }
\]
где $R(u)$ - оператор, отображающий пространство $U$ в пространство $Z$. Задача называется \textbf{корректно поставленной} (\textbf{корректной по Адамару}) на паре пространств $U$, $Z$, если:
\begin{enumerate}
    \item её решение $z$ существует для любого $u \in U$;
    \item её решение единственно для любого $u \in U$;
    \item решение задачи непрерывно зависит от входных данных, т.е.
    \[
        \rho_U
        \big(
        u_{\text{точное}},\,
        u_{\text{приближённое}}
        \big)
        \rightarrow 0
        \implies
        \rho_Z
        \big(
        R(u_{\text{точное}}),\,
        R(u_{\text{приближённое\\}})
        \big)
        \rightarrow 0
        \text{,}
    \]
    где $\rho_U$ и $\rho_Z$ - метрики на соответствующих пространствах.
\end{enumerate}

Обратные задачи нередко оказываются некорректными. Вопрос существования решения зачастую не рассматривается из-за его сложности, а также поскольку обратная задача формулируется исходя из целей обработки экспериментальных данных, что предполагает существование точного решения. Впрочем, наличие погрешности в доступных данных может сказаться на существовании классического решения. Чаще всего рассматриваются вопросы единственности и устойчивости решения, причём для большого класса задач можно доказать лишь единственность, в то время как устойчивость отсутствует. Последнее означает, что уменьшение погрешности измерения известных в задаче данных в общем случае не будет приводить к увеличению точности решения. Фундаментальный вклад в создание теории некорректно поставленных задач и методов их решения внесли работы А.\,Н.\,Тихонова \cite{arsenin, tikh_1, tikh_2}.

Важным классом обратных задач являются обратные задачи для уравнений эллиптического типа. Они находят применение в описании стационарных процессов, в частности, в описании стационарных полей в гравиразведке и медицине. Здесь следует упомянуть два класса задач, тесно связанных с темой данной работы. Первый из них - задачи Коши для уравнения Лапласа. В них требуется по значениям некоторой гармонической функции (например, потенциала электростатического поля) и её нормальной производной на части границы области определить значения этой функции внутри области. Иными словами, задача Коши для уравнения Лапласа зачастую есть задача продолжения потенциального поля внутрь области \cite{den}. Решение данной задачи единственно, но неустойчиво (пример Адамара \cite{sobolev}), поэтому класс рассматриваемых функций необходимо сузить, чтобы постановка была корректной. Например, можно рассматривать гармонические функции, ограниченные в области \cite{lavrentiev}.

Второй класс задач, который следует упомянуть - обратные задачи теории потенциала. Они состоят в поиске правой части в уравнении Пуассона по значениям потенциала в области, не содержащей источников поля. Источники поля обычно считаются сосредоточенными в некоторой ограниченной области $T$. Обратные задачи теории потенциала возникают во многих важных исследованиях, например, в гравиразведке полезных ископаемых и различных видах медицинской диагностики. Исследованию обратных задач теории потенциала посвящено большое число работ, см. например \cite{novikov, prilepko, sretenskiy}.

В обратных задачах для эллиптических уравнений плотность распределения рассматриваемой величины (массы, заряда) может быть сильно локализована. Тогда задача нахождения плотности этой величины в некоторой области есть задача восстановления точных источников. Так, ряд работ Е.В. Захарова \cite{zakh_1, zakh_2} посвящён задачам электроэнцефалографии, в которых источники поля - нейроны головного мозга - представляются в виде точечных электрических диполей. Такой подход весьма распространён в решении обратных задач электроэнцефалографии \cite{gnezditskiy}.

Целью выпускной квалификационной работы является разработка и программная реализация численного метода решения обратной задачи для уравнения Пуассона, состоящей в определении положения точечных источников внутри сферы в случае, когда потенциал на её поверхности равен нулю, а на сфере или её части известна нормальная производная потенциала. 