\newpage
\section{Введение}

Математическое моделирование различных процессов и явлений играет большую роль в современной науке и технике. Оно находит применение как в фундаментальных исследованиях в области физики, химии, биологии и/или других естественных наук, так и в прикладных задачах промышленности, медицины, экономики и т.д. \cite{samarskiy_models}. При этом часто естественно возникает условное разделение решаемых задач на два типа.

В первом случае известны внутренние параметры некой модели, и требуется на основе этих данных сделать количественные и качественные выводы о следующих из них явлениях. Например, по известным действующим на движущуюся материальную точку силам и её массе определить траекторию её движения. В таких задачах известна причина, и требуется определить следствие; их принято называть прямыми. Нередко подобные задачи встречаются в прикладном применении полученных ранее иными способами фундаментальных знаний в предметной области. Во втором случае, напротив, требуется по наблюдаемым внешним проявлениям определить внутренние параметры модели; по имеющемуся следствию установить причину. Задачи такого типа весьма характерны для научных исследований и называются обратными.

Обратные задачи часто связаны с обработкой экспериментальных данных. Это означает, что проблемы, вызванные принципиальными ограничениями проводимых экспериментов и доступными исследователям средствами, оказывают влияние и на принципы рассмотрения обратных задач. В связи с этим большое внимание в исследовании обратных задач уделяется понятию корректности постановки задачи, введённому в 1902\,г. французским математиком Жаком Адамаром \cite{hadamard}. Введём одну из формализаций этого понятия, следуя \cite{den}.

Пусть $u$ - известные нам данные, $z$ - неизвестная информация, которую предстоит определить (т.е. решение задачи). Предположим, что $u \in U$ и $z \in Z$, где $U$ и $Z$ - некоторые метрические пространства. Тогда формально соответствие между этими данными можно записать в виде
\[
z = R(u) \text{, } u \in U \text{, } z \in Z \text{, }
\]
где запись $R(u)$, строго говоря, не следует воспринимать как функцию (одному и тому же $u$ могут соответствовать несколько разных $z$). Задача называется \textbf{корректно поставленной}, \textbf{корректной по Адамару} (или просто \textbf{корректной}) на паре пространств $U$, $Z$, если:
\begin{enumerate}
    \item её решение $z$ существует для любого $u \in U$;
    \item её решение единственно для любого $u \in U$ (т.е. $R(u)$ является функцией);
    \item решение задачи непрерывно зависит от входных данных, т.е.
    \[
        \rho_U
        \big(
        u_{\text{точное}},\,
        u_{\text{приближённое}}
        \big)
        \rightarrow 0
        \implies
        \rho_Z
        \big(
        z_{\text{точное}},\,
        z_{\text{приближённое\\}}
        \big)
        \text{,}
    \]
    где $\rho_U$ и $\rho_Z$ - метрики на соответствующих пространствах.
\end{enumerate}

Обратные задачи часто оказываются некорректными. Существование решения обычно не является проблемой в силу того, что обратная задача формулируется исходя из целей обработки реальных экспериментальных данных. Впрочем, наличие погрешности в доступных данных может сказаться на существовании классического решения. Более частой проблемой является отсутствие единственности, и ещё чаще - отсутствие устойчивости решения. Последнее, например, означает, что уменьшение погрешности измерений исходных данных в общем случае не будет приводить к увеличению точности решения. Таким образом, вопрос корректности обратной задачи существенно влияет на её практическую ценность.

Данная работа посвящена задаче определения точечных источников для уравнения Пуассона. Рассматриваемые в задаче точечные электрические заряды полагаются одинаковыми и равными единице, поскольку задача одновременного определения формы тела и распределения плотности источников поля в нём заведомо некорректна \cite{den, prilepko}.