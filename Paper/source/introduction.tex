\newpage
\section{Введение}



В задачах математической физики важную роль играет понятие корректности постановки задачи (или просто корректности задачи). Оно было сформулировано в 1902\,г. французским математиком Жаком Адамаром. Для описания этого понятия формализуем постановку задачи.

Пусть $u$ - известные нам данные, $z$ - неизвестные, которые предстоит определить. Предполагаем, что $u \in U$ и $z \in Z$, где $U$ и $Z$ - метрические пространства. Тогда решение задачи можно записать в виде
\[
z = R(u) \text{,}
\]
и 


Задача называется корректной по Адамару, если:
